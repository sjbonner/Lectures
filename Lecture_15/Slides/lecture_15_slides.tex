\documentclass[xcolor=pdftex,dvipsnames,table,aspectratio=169]{beamer}
%\documentclass[xcolor=pdftex,dvipsnames,table,handout,aspectratio=169]{beamer}

%\setbeameroption{show notes}

\usepackage{bm,graphicx,multirow,amsmath,tikz} %fancybox,
\usepackage{color}%,textpos}
\usepackage[round]{natbib}
\usepackage[normalem]{ulem}
\usepackage{hyperref}
\usepackage{lastpage}
\usepackage{array}
\usepackage{color}
\usepackage{framed}
\usepackage{hyperref}

% Define Western colours
\definecolor{western}{rgb}{.306,.152,.524}
\definecolor{westerngray}{rgb}{.512,.508,.524}

%% Define BEAMER colours
\setbeamercolor{frametitle}{bg=western,fg=white}
\setbeamercolor{framesubtitle}{bg=western,fg=black}
\setbeamercolor{title}{fg=white,bg=western}
\setbeamercolor{author}{fg=white,bg=western}
\setbeamercolor{institute}{fg=white,bg=western}
\setbeamercolor{date}{fg=white,bg=western}

%% Set BEAMER fonts
\setbeamerfont{title}{shape=\bf}
\setbeamerfont{frametitle}{shape=\sc,size=\Large}
\setbeamerfont{framesubtitle}{shape=\sc,size=\Large}
\setbeamerfont{footline}{shape=\sc}

%% Define BEAMER toc
\setbeamercolor{section in toc}{fg=western}
\setbeamercolor{subsection in toc}{fg=westerngray}
\setbeamertemplate{sections/subsections in toc}[ball]

%% Define BEAMER background
\setbeamercolor{background canvas}{bg=white}

%% Define BEAMER footer
\setbeamertemplate{navigation symbols}{}
\setbeamercolor{footline}{fg=white,bg=western}
\setbeamertemplate{footline}{%
  \begin{beamercolorbox}[wd=\paperwidth]{footline}
    \vskip5pt

    \raisebox{.05in}{
      \scriptsize{\bf \insertshorttitle}
    }
    \hfill
    \raisebox{.05in}{
      \scriptsize{\bf \insertframenumber/\inserttotalframenumber} 
    }
    \hspace{5pt}

    \vskip5pt
  \end{beamercolorbox}
}

%% Define BLOCK environment
\setbeamercolor{block title}{fg=western}
\setbeamerfont{block title}{series=\bfseries}

%% Define ENUMERATE and ITEMIZE environements
\setbeamertemplate{itemize item}[ball]
\setbeamertemplate{enumerate item}[ball]
\setbeamercolor{item projected}{bg=western}

%% Define BEAMER toc
\setbeamercolor{sections/subsections in toc}{fg=blue!75}
\setbeamertemplate{sections/subsections in toc}[ball]

% %% Define SECTION openings
% \AtBeginSection[]{
%   \begin{frame}{\insertshorttitle}
%     \tableofcontents[currentsection,subsectionstyle=hide/hide/hide]
    
%   \end{frame}
% }

%% Define BEAMER frametitle
\addtobeamertemplate{frametitle}{
   \let\insertframetitle\insertsectionhead}{}
\addtobeamertemplate{frametitle}{
   \let\insertframesubtitle\insertsubsectionhead}{}


\makeatletter
  \CheckCommand*\beamer@checkframetitle{\@ifnextchar\bgroup\beamer@inlineframetitle{}}
  \renewcommand*\beamer@checkframetitle{\global\let\beamer@frametitle\relax\@ifnextchar\bgroup\beamer@inlineframetitle{}}
\makeatother

% Define counters for example and exercise
\newcounter{example}
\newcounter{exercise}

% Define example and exercise commands
\renewcommand{\example}
{\stepcounter{example}Example \lecturenum.\arabic{example}}
\newcommand{\examplectd}
{Example \lecturenum.\arabic{example}\ ctd}
\newcommand{\exercise}
{\stepcounter{exercise}Exercise \lecturenum.\arabic{exercise}}
\newcommand{\exercisectd}
{Exercise \lecturenum.\arabic{exercise}\ ctd}

\usepackage[normalem]{ulem}
\usepackage{enumitem}

\newcommand{\lecturenum}{15}

\title[SS2857]{Probability and Statistics I}
\subtitle{\lecturenum. Expected Values}

\date{}

%% Add logo
% \titlegraphic{\includegraphics[height=2cm]{../uwo_logo_reversed}}

%% Initialize R


\begin{document}

{
\setbeamertemplate{footline}{}
\setbeamercolor{background canvas}{bg=western}

\begin{frame}
  \addtocounter{framenumber}{-1}

  \maketitle
\end{frame}
}


\section{Expected Values}
\begin{frame}
  \frametitle{\invisible{Hello}}

  \begin{center}
    \Large{\textbf{4.2 Expected Values \sout{and Moment Generating Functions}}}
  \end{center}

  \bigskip

  \only<1>{
    \begin{center}
      \includegraphics[height=.5\textheight]{zurg}
    \end{center}
  }

  \only<2>{
    \begin{block}{Key Lesson}
      \begin{center}
        Replace 
        \begin{center}
        sums and pmfs (for discrete random variables) 
        \end{center}
        with
        \begin{center}
        integrals and pdfs (for continuous random variables).
        \end{center}
      \end{center}
    \end{block}
  }

\end{frame}

\begin{frame}
  
  \begin{block}{Mean}
    The expected or mean value of a continuous random variable with pdf $f(x)$ is
    \[
      \mu_X=E(X)=\int_{-\infty}^{\infty} xf(x) ~dx.
    \]
    The expected values exists if $\int_{-\infty}^{\infty} |x|f(x) ~dx< \infty$.
  \end{block}
\end{frame}

\begin{frame}

  \begin{block}{Variance}
    The variance of $X$ is
    \[
      \sigma^2_X=V(X)=E[(X-\mu)^2]=\int_{-\infty}^{\infty}(x-\mu)^2 f(x)~dx.
    \]
    The standard deviation is $\sigma_X=\sqrt{V(X)}$. 
    
    The variance and standard deviation exist if $\int_{-\infty}^{\infty} (x-\mu)^2f(x) ~dx < \infty$.
  \end{block}
\end{frame}

\begin{frame}

  \begin{block}{Variance}
    There is a shortcut formula for the variance. For *any* random variable $X$ (discrete or continuous) the variance can be computed as
    $$
    \sigma^2_X=E(X^2)-E(X)^2=E(X^2)-\mu_X^2.
    $$
  \end{block}
\end{frame}

\begin{frame}
  \begin{block}{\example}
    Consider the random variable, $X$, from Example 14.1. This random variable represents the distance that an object dropped from a height of 1~m falls in a randomly selected time between 0 and 1 second on the home planet of Emperor Zurg where the force of gravity is only 2~m/s$^2$. The pdf and cdf are
    \begin{scriptsize}
      \[
        f(x)=\left\{
          \begin{array}{ll}
            0 \leq 0\\
            2x & 0 < x < 1\\
          0 & 1 \leq x
          \end{array}
        \right.
        \mbox{ and }
        F(x)=\int_{-\infty}^x f(u)~du=
        \left\{
          \begin{array}{ll}
            0 & x \leq 0\\
            x^2 & 0 < x < 1\\
            1 & 1 \leq x
          \end{array}
        \right.
      \]
    \end{scriptsize}
  \end{block}
  
  \begin{enumerate}[label=\alph*),start=1]
  \item Find the mean of $X$.
  \item Compute the variance of $X$.
  \item Provide an interpretation for the mean.
  \end{enumerate}
  
\end{frame}

\begin{frame}
  
  \begin{block}{Mean}
    Let $Y=h(X)$ for some function $h(\cdot)$. Then:
    \[
      \mu_Y=E(Y)=\int_{-\infty}^{\infty} h(x)f(x) ~dx
      \mbox{ and }
      \sigma^2_Y=V(Y)=\int_{-\infty}^{\infty}(h(y)-\mu_Y)^2 f(x)~dx.
    \]
    
    \bigskip
    
    If $Y$ is a linear function of $X$, $Y=aX+b$, then
    $$
    \mu_Y=a\mu_X+b
       \mbox{ and }
    \sigma^2_Y=a^2\sigma^2_X.
    $$
  \end{block}
\end{frame}

\begin{frame}
  \begin{block}{\examplectd}
    Consider the random variable, $X$, from Exmple 14.1. This random variable represents the distance that an object dropped from a height of 1~m falls in a randomly selected time between 0 and 1 second on the home planet of Emperor Zurg where the force of gravity is only 2~m/s$^2$. The pdf and cdf are
    \begin{scriptsize}
      \[
        f(x)=\left\{
          \begin{array}{ll}
            0 & x \leq 0\\
            2x & 0 < x < 1\\
          0 & 1 \leq x
          \end{array}
        \right.
        \mbox{ and }
        F(x)=\int_{-\infty}^x f(u)~du=
        \left\{
          \begin{array}{ll}
            0 & x \leq 0\\
            x^2 & 0 < x < 1\\
            1 & 1 \leq x
          \end{array}
        \right.
      \]
    \end{scriptsize}
  \end{block}
  
  \begin{enumerate}[label=\alph*),start=4]
  \item Let $Y$ be the distance traveled in inches not metres ($Y=39.37X$). Find the mean and variance of $Y$.
  \item Let $Z=X^2$. Find the mean and variance of $Z$. 
  \end{enumerate}
  
\end{frame}
\begin{frame}
  \begin{block}{Approximating the Mean and Variance}
    The proposition on page 174 (called the delta method) suggests that if $Y=h(X)$ for some function $h(x)$ which is differentiable and has non-zero deriviative at $\mu=E(X)$ then
\[
E(Y)\approx h(\mu)
\]
and
\[
V(Y)\approx h'(\mu)^2 V(X)
\]
if ``the variance of $X$ is small''.

\bigskip

\begin{center}
  \textbf{This is too vague!}
  We will discuss the full result later.
\end{center}
  \end{block}
\end{frame}

\begin{frame}
  \begin{center}
    \Large{\textbf{Questions?}}
  \end{center}
\end{frame}

\begin{frame}

\begin{block}{\exercise}
Suppose that the random variable $X$ has pdf
$$
f(x)=\frac{3}{4}\left[2x-x^2\right],\quad 0 \leq x \leq 2.
$$

\begin{enumerate}[label=\alph*),start=1]
\item Confirm that $f(x)$ is a valid pdf.
\item Find the mean and variance of $X$.
\item Find the mean and variance of $Y=3X+2$.
\item Find the mean and variance of $Z=X^2$.
\end{enumerate}
\end{block}
\end{frame}
\end{document}
