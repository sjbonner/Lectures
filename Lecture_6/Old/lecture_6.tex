%\documentclass[addpoints]{exam}
\documentclass[addpoints,answers]{exam}

\usepackage[margin=1in]{geometry}
\usepackage{enumerate,amsmath,graphicx}

\usepackage{arev}
\usepackage[T1]{fontenc}

\begin{document}
\begin{center}
  \Large{\textsc{\textbf{SS2857 Probability and Statistics I}}}\\
  \Large{Fall 2021}

  \bigskip
  
  \large{Chapter 2 Summary Exercise: \textsc{Hardy-Weinberg Equilibrium}}

\end{center}

\section{Introduction}

As a real-world example of some of the material fro Chapter 2, we are going to consider a problem in population genetics: modelling frequencies of alleles in a population with random mating. 

For those of you not familiar with genetics, here is a very brief (and simplified) primer. A gene is a location on a chromosome within an organism and the alleles for a gene are the possible variants of the DNA sequence that can occur at that location. Diploid organisms, which we will consider, possess two copies of each chromosome and hence two alleles for each gene (barring the genes on the sex chromosomes which may be different, but we will ignore these). Humans are diploid organisms with a single sex chromosome so that men have only one copy of some genes. The genotype of an individual is then determined by the pair of alleles that they inherit from their mother and father. For the genes that are not on the sex chromosome, the phenotype is determined by how these alleles express themselves. An allele is said to be dominant if its phenotype is expressed when at least one copy of the allele is present. It is said to be recessive if its phenotype is expressed only when both copies of the allele are the same.

Imagine a situation in which individuals can mate only with other individuals in the same generation. I.e., we start with a parent population, mating occurs, and this creates an offspring population. The offspring then go on to mate and produce new offspring, but at no point does an individual mate with an individual from a different generation. This doens't happen with humans but does happen with some organisms that mate on an annual cycle. We say that random mating occurs if two conditions hold:
\begin{enumerate}
\item the two alleles an offspring inherits from each parent are independent, and 
\item the probability that each allele takes a specific form, say $A$, is equal to the proportion of allele $A$ in the parent population.
\end{enumerate}

A common example is the process for determing eye colour and the alleles for brown versus blue eyes. The allele for brown eyes is dominant to the allele for blue eyes. A person's eyes will be brown (their phenotype) if they have two copies of the brown eye allele or one copy of the brown eye allele and one copy of the blue eye allele (their genotype). A person's eyes will be blue only if they have two copies of the blue eye allele. If these were the only two alleles, then we would say that the allele for brown eyes was dominant and the allele for blue eyes was recessive. Symbolically, we can let $A$ and $a$ represent the alleles for brown and blue eyes, respectively. The capital letter indicates that the first allele is dominant. The possible genotypes and their correspoding phenotypes are
\begin{center}
  \begin{tabular}{ll}
    Genotype & Phenotype\\
    \hline
    AA & Brown\\
    Aa & Brown\\
    aa & Blue
  \end{tabular}
\end{center}
Notice that the genotypes are unordered. I.e., when we talk about genotype we don't distinguish which allele came from the mother and which came from the father. 

\newpage

\section{Questions}

\begin{questions}

\question
   A population contains $n_{AA}$ people with two alleles for brown eyes, $n_{Aa}$ with one allele for brown eyes, one for blue eyes, and $n_{aa}$ with two alleles for blue eyes, and $n=n_{AA}+n{Aa}+n_{aa}$ be the total population size. Suppose that we sample a person at random and then sample an allele at random.
  
  \begin{enumerate}[a)]
  \item What is the probability that the person has brown eyes?
  \item What is the probability that the sampled allele is the allele for brown eyes? 
  \item What is the probability that the sampled person has brown eyes given that the sampled allele is for blue eyes?
  \item What do these probabilities mean?
  \end{enumerate}
  
  \begin{solution}
  \begin{enumerate}[a)]
  \item Let $B$ be the event that the sampled person has brown eyes. Saying that we "sample a person at random" implies that the outcomes of choosing any individual from the population is equally likely. In this case, the probability is:
  $$
  P(B)=\frac{n(B)}{N}=\frac{n_{AA}+n_{Aa}}{n}.
  $$
  
   \item Let $G_1$ be the event that the selected individual has genotype $AA$, $G_2$ the event they have genotype $Aa$, and $G_3$ the even they have genotype $aa$. Note that $G_1$, $G_2$, and $G_3$ partition the sample space. Then
  $$
  P(G_1)=\frac{n_{AA}}{n},\quad P(G_2)=\frac{n_{Aa}}{n},\quad P(G_3)=\frac{n_{a}}{n}.
  $$
  Let $A$ be the event that we select the allele for brown eyes. Then, 
  $$
  P(A|G_1)=1, \quad P(A|G_2)=.5, \quad P(A|G_3)=0.
  $$
  Applying the law of total probability,
  $$
  \begin{aligned}
  P(A)
  &=P(A|G_1)P(G_1) + P(A|G_2)P(G_2) + P(A|G_3)P(G_3)\\
  &=\frac{n_{AA}}{n} \cdot 1 + \frac{n_{Aa}}{n} \cdot .5 + \frac{n_{a}}{n} \cdot 0\\
  &=\frac{n_{AA} + .5n_{Aa}}{n}\\
  &=\frac{2n_{AA} + n_{Aa}}{2n}
  \end{aligned}
  $$
  We will denote this value by $p$. Then the probability that the selected allele is the allele for blue eyes is $1-p$. Note that these are simply equal to the proportions of alleles for brown eyes and blues out of all allelles in the population.

  \item This question asks us for the probability of the event $B|A'$. Note that $B=G_1 \cup G_2$ and also that $G_1$ and $G_2$ are disjoint. Then
  $$
  \begin{aligned}
  P(B|A')
  &=P(G_1 \cup G_2|A')\\
  &=P(G_1|A') + P(G_2|A').
  \end{aligned}
  $$
  Applying Bayes' rule
  $$
  P(G_1|A')=P(A'|G_1)P(G_1)/P(A')=\frac{0(n_{AA}/n)}{1-(2n_{AA} + n_{Aa})/2n}=0
  $$
  and
  $$
  P(G_2|A')=P(A'|G_2)P(G_2)/P(A')=\frac{.5(n_{Aa}/n)}{1-(2n_{AA} + n_{Aa})/2n}=\frac{n_{Aa}}{n_{Aa}+2 n_{aa}}.
  $$
  Hence, 
  $$
  P(B|A')=\frac{n_{Aa}}{n_{Aa}+2 n_{aa}}.
  $$
  \item The probability that the sampled allele is $A$ is $p$. This means that if we were to repeat this process of selecting an individual at random and then choosing one of their alleles many, many times then the proportion of times we draw allele $A$ will be very close to $p$ and will get closer and closer as the sample size increases. 
  \end{enumerate}
  \end{solution}
    
    \question
    Let the ordered pair $(m,f)$ denote the alleles an offspring inherits from its parents. 
  
  \begin{enumerate}[a)]
  \item What are the possible outcomes?
  \item What are the genotypes and phenotypes associate with each outcome?
  \end{enumerate}
  
    \begin{solution}
    \begin{enumerate}[a)]
    \item There are four possible outcomes in total: $(A,A)$, $(A,a)$, $(a,A)$, and $(a,a)$.
    \item The following table indicates the genotype and phenotype associate with each outcome:
    \begin{center}
    \begin{tabular}{ccc}
    Outcome & Genotype & Phenotype\\
    \hline
    $(A,A)$ & AA & Brown\\
    $(A,a)$ & Aa & Brown\\
    $(a,A)$ & Aa & Brown\\
    $(a,a)$ & aa & Blue
    \end{tabular}
    \end{center}
    \end{enumerate}
    \end{solution}
    
    \question
    Let $M$ be the event that the offspring inherits the allele for brown eyes from its mother and $M'$ the event it inherits the allele for blue eyes from its mother. Let  $F$ be the events that the offspring inherits the allele for brown eyes from its father and $F'$ the event it inherits the allele for blue eyes its father.

  \begin{enumerate}[a)]
  \item What are the outcomes within each event?
  \item Which pairs of the events $M$, $M'$, $F$, and $F'$ are mutually exclusive and which are not?
  \item Define the events associated with each genotype in terms of $M$, $M'$, $F$, and $F'$.
  \end{enumerate}
  
    \begin{solution}
    \begin{enumerate}[a)]
    \item Each of these four events is associated with two of the outcomes in the sample space:
      \begin{center}
        \begin{tabular}{ll}
          Event & Outcomes\\
          \hline
          $M$ & $(A,A)$, $(A,a)$\\
          $M'$ & $(a,A)$, $(a,a)$\\
          $F$ & $(A,A)$, $(a,A)$\\
          $F'$ & $(A,a)$, $(a,a)$
        \end{tabular}
      \end{center}
    \item From the table, we can easily see that $M$ and $M'$ and $F$ and $F'$ are mutually exclusive. This is also true by the definition of the complement. Any other pair has a single event in common. 
      \item The possible genotypes and their associated events are
      \begin{center}
        \begin{tabular}{ll}
          Genotype & Events\\
          \hline
          $AA$ & $M \cap F$\\
          $Aa$ & $(M \cap F') \cup (M' \cap F)$\\
          $aa$ & $M' \cap F'$\\
        \end{tabular}
      \end{center}
    \end{enumerate}
    \end{solution}
    
    \question
   The activity at the start of class is intended to simulate random mating. Biologists say that a population is mating at random if two conditions hold:
  \begin{enumerate}
  \item the two alleles an offspring inherits from each parent are independent, and 
  \item the probability that each allele takes a specific form, say $A$, is equal to the proportion of allele $A$ in the parent population.
  \end{enumerate}

  \medskip

  Assuming random mating:
  \begin{enumerate}[a)]
  \item Compute the probabilities for the events $M$, $M'$, $F$, $F'$.  
  \item Compute the probability of the possible genotypes for the offspring. 
  \item What is the probability that an offspring has brown eyes? 
  \item What is the probability that the offspring has blue eyes?
  \end{enumerate}
  
  \begin{solution}
  \begin{enumerate}[a)]
  \item The second condition of random mating implies that $P(M)=p$, $P(M')=1-p$, $P(F)=p$, and $P(F')=p$. 
  
  \item Given independence of the events $M$ and $F$ we have:
      \[
        P(M \cap F)=P(M) P(F)=p^2. 
      \]
      Similarly, the probability of the genotype $aa$ is
      \[
        P(M' \cap F')=P(M') P(F')=(1-p)^2. 
      \]
      By subtraction, the probability of the genotype $Aa$ is
      \[
        P((M \cap F') \cup (M' \cap F))=1-P(M \cap F)-P(M' \cap F')=1-p^2-(1-p)^2=2p(1-p).
      \]
      Alternatively, since the events $(M \cap F')$ and $(M' \cap F)$ are mutually exclusive and the pairs $M$ and $F'$ and $M'$ and $F$ are independent
      \[
        P((M \cap F') \cup (M' \cap F))=P(M \cap F') + P(M' \cap F)=2p(1-p).
      \]
      
      \item The offspring has brown eyes if it inherits one or two copies of the dominant allele $A$. This is described by the event $(M \cap F) \cup (M \cap F') \cup (M' \cap F)$. These events are all mutually exclusive, so the probability is
      \[
         P(M \cap F) + P(M \cap F') + P(M' \cap F)=p^2 + 2p(1-p).
       \]
       \item By subtraction, the probability that the offspring has blue eyes is 
       \[
         1-(p^2 + 2p(1-p)=(1-p)^2.
       \]
       
       Alternatively, the event that the offspring has blue eyes is $(M' \cap F')$ which has probability $(1-p)^2$. The probability that the offspring has brown eyes is then
       \[
         1-(1-p)^2=p^2 + 2p(1-p).
       \]
    \end{enumerate}
    \end{solution}
    
    \question
     Suppose that a population undergoes random mating, we randomly select one offspring from the new population, and then we randomly select one of the two alleles from this offspring.

  \medskip
  
  \begin{enumerate}[a)]
  \item What is the probability that the sampled allele is $A$? 
  \item What is the probability that it is $a$?
  \item What is the probability that the offspring has brown eyes?
  \item What is the probability that the offspring has blue eyes?
  \end{enumerate}

    \begin{solution}
    \begin{enumerate}[a)]
    \item Let $O_1$ be the event that we sample allele $A$. Then
      \[
      \begin{aligned}
        P(O_1)&=P(M \cap F) + 1/2 P(M \cap F') + 1/2 P(M' \cap F) + 0 P(M' \cap F')\\
        &=p^2 + 2(p(1-p)/2) + 0 (1-p)^2 = p.
        \end{aligned}
      \]
      \item Similarly, the probability that we obeserve $a$ is $P(O_1')=(1-p)$. This means that the probabilities of sampling the different types of alleles in the offspring population is exactly the same as in the parent population. This is what the word equilibrium refers to. The probabilities for the alleles remains constant from one generation to the next. 
      \item Let $O_2$ be the event that the offspring has brown eyes. Then
      $$
      P(M \cap F) + P(M \cap F') + P(M' \cap F)=p^2 + 2p(1-p)=2p-p^2.
      $$
      \item The probability that the offspring has blue eyes is
      $$
      P(O_2')=1-P(O_1)=(1-p)^2.
      $$
      \end{enumerate}
    \end{solution}
    
    \question Suggest two ways that the assumption of random mating might be violated.

    \begin{solution}
      The first assumption would be violated if there is sexual selection -- mates choose each other based on their phenotypes. The second assumption would be violated if individuals with one allele are more likely to mate than individuals with the other allele. This is called fitness. 
    \end{solution}
    
\end{questions}
\end{document}

%%% Local Variables:
%%% mode: latex
%%% TeX-master: t
%%% End:
